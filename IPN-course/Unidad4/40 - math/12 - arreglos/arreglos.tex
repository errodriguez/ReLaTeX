\documentclass[12pt,letterpaper]{article}
\usepackage[utf8]{inputenc}
\author{Curso de \LaTeX}
\title{Arreglos}
\begin{document}
\maketitle

El entorno \texttt{array} (no hay que confundirlo con el paquete \texttt{array}) es similar al entorno \texttt{tabular} con la diferencia de que puede ser incluido en el entorno matemático. Por ejemplo:

\[
\left(
\begin{array}{clrr}
      a+b+c & uv & x-y & 27 \\
       x+y  & w  & +z  & 363 \\
       a & b & c & d
\end{array}
\right)
\]

\[
\left(
\begin{array}{clrr}
	a+b+c & uv & x-y & 27 \\
	x+y  & w  & +z  & 363 \\
	a & b & c & d
\end{array}
\right), \begin{array}{rcrl}
	2x & + & 4y & = 7\\
 	  x & + & 3y & = 6
\end{array}
\]

Como vemos, el comando \textbackslash\textbackslash\ también es utilizado para ingresar saltos de línea en el arreglo.

Los arreglos pueden ser utilizados en el mismo entorno que el resto de las expresiones matemáticas vistas hasta ahora:

\begin{equation}
X = \left[
		\begin{array}{ccc}
			x_1 & x_2 & \ldots \\
			x_3 & x_4 & \ldots \\
			\vdots & \vdots & \ddots
		\end{array}
	\right]
\end{equation}
\end{document}