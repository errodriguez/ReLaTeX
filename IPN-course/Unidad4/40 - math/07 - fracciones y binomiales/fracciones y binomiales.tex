\documentclass[12pt,letterpaper]{article}
\usepackage[utf8]{inputenc}
\usepackage{amsmath} %Para soportar el comando \binom
\author{Curso de \LaTeX}
\title{Fracciones y binomiales}
\begin{document}
\maketitle

Una fracción es creada utilizando el comando \textbackslash \texttt{frac\{númerador\}\{denominador\}}. (Para aquellos que no lo recuerden, esos son la parte de arriba y la parte de abajo de la fracción respectivamente). Por ejemplo:

\begin{equation}
\frac{4}{5}
\end{equation}

\begin{equation}
\frac{x + 2}{y^2}
\end{equation}

El coeficiente binomial puede ser representado utilizando el comando \textbackslash \texttt{choose}:

\begin{displaymath}
	\frac{n!}{k!(n-k)!} = {n \choose k}
\end{displaymath}

Una alternativa para representar el coeficiente binomial es utilizar el comando \textbackslash \texttt{binom}\footnote{Requiere del paquete \texttt{amsmath}}:

\begin{displaymath}
\frac{n!}{k!(n-k)!} = \binom{n}{k}
\end{displaymath}

También podemos representar fracciones utilizando el comando \textbackslash\texttt{over}:

\begin{equation}
x + 2 \over y^2
\end{equation}

\begin{equation}
{n! \over k!(n-k)!} = {n \choose k}
\end{equation}

Puedes además embeber una fracción dentro de otra:

\begin{displaymath}
\frac{\frac{1}{x}+\frac{1}{y}}{y-z}
\end{displaymath}

Noten que al escribir una fracción $ \frac{2}{7} $ dentro del texto, esta es significativamente más pequeña a si la escribiéramos en dentro de un entorno de visualización especial.
\end{document}