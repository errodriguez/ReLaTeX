\documentclass[12pt,letterpaper]{article}
\usepackage[utf8]{inputenc}
\usepackage{amsmath}
\title{Ecuaciones con amsmath}
\author{Curso de \LaTeX}
\date{}
\begin{document}
\maketitle

El paquete \texttt{amsmath} nos provee de otros entornos para escribir nuestras ecuaciones. 

\section{equation*}

El entorno \texttt{equation*} nos permite escribir ecuaciones no enumeradas (de la misma forma que \texttt{displaymath}):

\begin{equation*}
  w = x_{i} + y_{i}\mu
\end{equation*}

\section{align y align*}

Los entornos \texttt{align} y \texttt{align*} son utilizados para escribir ecuaciones de varios renglones conservando la alineación de la parte derecha de la ecuación:
\begin{align}
 f(x) &= (x+a)(x+b) \\
 &= x^2 + (a+b)x + ab
\end{align}

\begin{align*}
	f(x) &= (x+a)(x+b) \\
	&= x^2 + (a+b)x + ab
\end{align*}

Tal y como con las matrices y tablas,  \textbackslash\textbackslash\ es utilizado para introducir un salto de línea, mientras que el caracter \& es utilizado para separar las dos partes de la ecuación, de manera similar a como se separan las columnas en una tabla.
\end{document}
