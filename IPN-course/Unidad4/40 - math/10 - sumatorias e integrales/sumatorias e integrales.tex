\documentclass[12pt,letterpaper]{article}
\usepackage[utf8]{inputenc}
\usepackage[hidelinks]{hyperref}
\author{Curso de \LaTeX}
\title{Sumatorias e integrales}
\begin{document}
\maketitle

Los comandos \textbackslash \texttt{sum} y \textbackslash \texttt{int} insertan los símbolos de sumatoria e integral respectivamente, con límites especificados utilizando el signo de intercalación (\^\space) y el guión bajo (\_). La notación típica de la sumatoria es:

\begin{equation}
\sum_{i=1}^{10} t_i
\end{equation}

Al escribir la sumatoria en línea como el texto, ésta se compacta para aparecer en una sola línea, ejemplo: $ \sum_{i=1}^{10} t_i $. Por otro lado, en la integral los límites siguen la misma notación:

\begin{equation}
\int_a^b f(x) dx
\end{equation}

Es también importante representar la integración con una d vertical, que en el entorno matemático se obtiene a través del comando \textbackslash mathrm\{\}, y con un pequeño espacio que la separe del integrando, lo cual es conseguido con el comando que combina la diagonal invertida con la coma (\textbackslash ,), tal y como se muestra a continuación:

\begin{equation}
\int_0^\infty e^{-x}\,\mathrm{d}x
\end{equation}

Hay un gran número de comandos que definen sus límites de manera similar a las sumatorias e integrales, por ejemplo:

\begin{displaymath}
\oint_1^n
\end{displaymath}

Puedes consultar la lista completa consulta en la guía: \url{https://en.wikibooks.org/wiki/LaTeX/Mathematics#Sums\_and\_integrals}
\end{document}