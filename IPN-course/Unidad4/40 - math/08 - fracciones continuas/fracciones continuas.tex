\documentclass[12pt,letterpaper]{article}
\usepackage[utf8]{inputenc}
\usepackage{amsmath}
\author{Curso de \LaTeX}
\title{Fracciones continuas}
\begin{document}
\maketitle

Si queremos que una fracción en línea con el texto no se compacte verticalmente, podemos usar los comandos \texttt{\textbackslash dfrac} y \texttt{\textbackslash cfrac}\footnote{Ambos requieren el paquete \texttt{amsmath}}, por ejemplo con el primer comando obtenemos $ \dfrac{2}{7} $.

También podemos utilizar este comando para prevenir la compactación de fracciones en un entorno separado del texto:

\begin{displaymath}
	\frac{\dfrac{1}{x}+\dfrac{1}{y}}{y-z}
\end{displaymath}

Pero, las fracciones continuas se deben escribir utilizando el comando \texttt{\textbackslash cfrac}:

\begin{equation}
  x = a_0 + \cfrac{1}{a_1
          + \cfrac{1}{a_2
          + \cfrac{1}{a_3 + \cfrac{1}{a_4} } } }
\end{equation}

Ya que el comando \texttt{\textbackslash dfrac} no mantiene un espaciado vertical uniforme:

\begin{equation}
	x = a_0 + \dfrac{1}{a_1
		+ \dfrac{1}{a_2
		+ \dfrac{1}{a_3 + \dfrac{1}{a_4} } } }
\end{equation}

\end{document}