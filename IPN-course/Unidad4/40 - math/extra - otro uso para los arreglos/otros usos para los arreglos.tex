\documentclass[12pt,letterpaper]{article}
\usepackage[utf8]{inputenc}
\usepackage{amsmath} %Para soportar el entorno cases
\author{Tu nombre}
\title{Otro uso para los arreglos}
\begin{document}
\maketitle

Podemos utilizar el comando \textbackslash\texttt{array} combinándolo con los delimitadores para escribir funciones definidas por partes:

\[ 
z = \left\{
              \begin{array}{ll}
                   1 & (x>0)\\
                   0 & (x<0)
              \end{array}
     \right.
\] %El comando \right. produce un delimitador de cierre invisible, de forma que se delimitará correctamente el contenido sin mostrar la llave al final

Este mismo efecto puede lograrse directamente con el entorno \textbackslash\texttt{cases}\footnote{Requiere el paquete \texttt{amsmath}}:

\[
 u(x) =
  \begin{cases}
   \exp{x} & \text{Si } x\geq0 \\
   1       & \text{Si } x < 0
  \end{cases}
\]
\end{document}