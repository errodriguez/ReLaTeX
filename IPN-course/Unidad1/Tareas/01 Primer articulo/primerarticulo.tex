%Preámbulo del documento 
\documentclass{article}
\title{LaTeX: an introduction}
\author{Mike Unwalla \and Eduardo Rodríguez Ávila \thanks{SEPI, UPIICSA-IPN,  \texttt{errodriguez@ipn.mx} }}
\date{Spring 2006} 
%Cuerpo del documento:
\begin{document}
	\maketitle
	\begin{abstract}
	LaTeX is a powerful typesetting system, used for 
	producing scientific and mathematical 
	documents of high typographic quality. Unlike
	WYSIWYG tools such as FrameMaker and Word,
	it uses plain text files that contain formatting
	commands. It's big, open source, stable and used
	by many technical publishing companies. It's
	also relatively unknown in the technical writing
	community. This article overviews LaTeX, and
	directs you to sources of information.
	\end{abstract}
	
	\section{History}
	
	Donald E Knuth (www-cs-faculty.stanford.edu/~knuth) 
	designed a typesetting program
	called TeX in the 1970s especially for complex
	mathematical text. LaTeX is a macro package
	that allows authors to use TeX easily, and uses
	TeX as its formatting engine. It is available for
	most operating systems; for example, you can
	use it on low-specification PCs and Macs, as well
	as on powerful UNIX and VMS systems. There
	are many different implementations of LaTeX.
	
	\section{Who uses it?}
	
	I first came across LaTeX in 1992, when fellow
	students were using it to write academic papers
	and theses. These days, it is widely used in
	the technical publishing industry for academic
	journals, particularly by mathematicians,
	physicists and other people who have complex
	notational requirements. For example, Elsevier,
	IEEE and the Royal Society all provide author
	guidelines for people who use LaTeX. One of
	my clients uses LaTeX to produce software
	documentation (see pages 18-20 of the Autumn
	2005 Communicator) and so I needed to learn it.
\end{document}