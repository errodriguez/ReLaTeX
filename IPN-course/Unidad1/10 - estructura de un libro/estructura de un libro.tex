\documentclass[letterpaper, 11 pt,oneside]{book}
\usepackage[utf8]{inputenc}
\usepackage[spanish]{babel} %Para traducir los encabezados al español automáticamente
\title{Ciudades del mundo}
\author{Curso de \LaTeX}
\begin{document}
\maketitle
\part{Ciudades europeas}
    \chapter{Ciudades históricas de Europa}
   	Europa es un continente grande y rico en historia y cultura, con miles de lugares para visitar, disfrutar, descubrir, aprender, donde descansar o divertirse. Las ciudades europeas tienen siglos de historia, de luchas y enfrentamientos por el poder las cuales han dejado huella en el tiempo y en maravillosas y auténticas obras de arte cultural. Europa tiene impresionantes castillos, imponentes catedrales, palacios señoriales... Por todo lo anterior es muy difícil hacer una lista de 10 ciudades ya que cada ciudad tiene algo especial y con encanto. Por un lado, no es lo mismo que una ciudad sea objetivamente hermosa por su situación, su encanto general y su personal urbanismo o porque tenga hasta media docena de edificios singulares de especial galanura y ello le otorgue mucha categoría. Por otro, los ciudadanos siempre opinarán que la suya es la ciudad más bella, la más limpia, la más divertida, donde mejor se come y se ríe, la más simpática. Entonces, existe un millar de razones, más o menos, para colocar a cualquiera de ellas en un listado de preferencia estética o afectuosa.
   	
	    \section{Viena}
        Viena es una ciudad de Europa Central situada a orillas del Danubio, en el valle de los Bosques de Viena, al pie de las primeras estribaciones de los Alpes. Capital de Austria, así como uno de sus nueve Estados federados (Bundesland Wien).\\
 
        Está rodeada por el Estado federado de Baja Austria. Con una población de 1.712.903 habitantes (2010), Viena es la mayor ciudad, centro cultural y político de Austria. El área metropolitana cuenta con 2,4 millones de habitantes, población similar a la de la ciudad en 1914. El idioma oficial es el alemán.\\
 
        \section{Sevilla}
        Sevilla es un municipio y ciudad española, capital de la provincia homónima y de la comunidad autónoma de Andalucía. Ostenta los títulos de "Muy Noble, Muy Leal, Muy Heroica, Invicta y Mariana Ciudad de Sevilla". Sevilla contaba en 2010 con 704.198 habitantes (INE, 2010), siendo la cuarta    ciudad de España por población después de Madrid, Barcelona y Valencia.

	        \subsection*{Catedral}
            La Catedral de Sevilla es la catedral gótica más extensa del mundo y uno de los mayores templos cristianos en cuanto a tamaño, del mundo. Fue declarada por la UNESCO Patrimonio de la humanidad en 1987.
 
            \subsection*{Giralda}
            La Giralda es el campanario de la Catedral de Sevilla y la torre más representativa de la ciudad. Mide 104 metros de altura y fue iniciada en el siglo XII como alminar almohade de la mezquita mayor hoy desaparecida, a imagen y semejanza del alminar de la mezquita Kutubia de Marrakech (Marruecos), no obstante su coronación renacentista y campanario, obra de Hernán Ruiz, fue construida entre 1558 y 1568 por encargo del cabildo catedralicio.
\part{Ciudades asiáticas}
 	\chapter{Ciudades históricas de Asia}
 	Las principales ciudades de Asia son destinos maravillosos, modernos y vanguardistas que ofrecen hermosos rincones y lugares encantadores por conocer. Sin duda alguna cada una de estas ciudades cuenta con un atractivo turístico único, una historia milenaria y muchos establecimientos.
 	
 	Probablemente muchos relacionan los países y ciudades de Asia con un pasado milenario y tienen razón. Pero además de cultura e historia, el continente asiático cuenta con infinidad de ciudades impresionantemente modernas y tecnológicas.
 	
	 	\section{Pekín}
	 	Corazón cultural, político y social de China, Pekín es una de las ciudades con mayor atractivo turístico del mundo. Ocupa en la actualidad, con más de 20.000.000 de habitantes, el séptimo lugar en población mundial. Visita obligada en esta ciudad es Ciudad Prohibida y su maravilloso Palacio de la Suprema Armonía.
		\section{Tokio}	
 		La majestuosa capital de Japón es una inmensa urbe con impresionantes edificios que generan hermosos paisajes. Tokio Cuenta con una gran cantidad de lugares donde ir de compras y muchos lugares para ir a divertirse durante la noche. Es ideal para todos los gustos pues cuenta con templos antiguos, lugares tecnológicos totalmente increíbles y estupendos lugares para comer.
 		\section{Singapur}
 		Una ciudad encantadora y mágica llena de luces y lugares que visitar. Singapur, es sin duda un destino que no debes perderte. Los Jardines de la Bahía, el Templo Buddha Tooth Relic y el Jardín Botánico, son algunas de las vistas obligadas de la ciudad.
 		\section{Shanghái}
 		La bella Shanghái en China es una de las ciudades más interesantes y modernas que encontramos entre las principales ciudades de Asia. Hoy en día cuenta con sorprendentes edificaciones, una gran cantidad de atracciones, vistas maravillosas y excelentes restaurantes para conocer y disfrutar de su gastronomía.
\end{document}