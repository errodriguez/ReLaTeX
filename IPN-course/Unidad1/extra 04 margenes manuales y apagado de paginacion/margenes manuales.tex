\documentclass[12pt,letterpaper]{book}
\usepackage[utf8]{inputenc}
\usepackage{lipsum}
\author{Curso de LaTeX}
\title{Personalizar los márgenes manualmente}
%Modificar los margenes del documento manualmente
\addtolength{\oddsidemargin}{-.1cm} %Reducir margen para páginas impares
\addtolength{\evensidemargin}{-2cm} %Reducir margen para páginas pares
\addtolength{\textwidth}{2cm} %Agregar al ancho del texto
\addtolength{\topmargin}{-1.5cm} %Reducir margen superior
\addtolength{\textheight}{2cm} %Agregar al alto del texto
%Apagar la paginación del documento, para que no se muestre la paginación de los capítulos introductorios
\pagestyle{empty}
\begin{document}
\frontmatter
\maketitle
\chapter*{Lorem ipsum dolor}
% Debido a que el comando chapter hace que se vuelva a imprimir la paginación, hay que volver a apagarla con el comando:
\thispagestyle{empty}
\lipsum
\tableofcontents
% Con el comando tableofcontents ocurre lo mismo que con chapter, así que hay que aplicar nuevamente la corrección:
\thispagestyle{empty}
\mainmatter
% Para el contenido principal volvemos al estilo de paginación por defecto:
\pagestyle{plain}
 \chapter{Volutpat leo habitant}
\lipsum
 \chapter{Adipiscing diam euismod}
\lipsum[1]
 \section{Curabitur quam}
\lipsum[2]
\end{document}