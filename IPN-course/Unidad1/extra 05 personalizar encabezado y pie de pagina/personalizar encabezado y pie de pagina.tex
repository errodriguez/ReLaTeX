\documentclass[12pt,letterpaper,oneside]{book}
\usepackage[utf8]{inputenc}
\usepackage{lipsum}
\author{Curso de LaTeX}
\title{Personalizar encabezado y pie de página}
%Modificar los margenes del documento manualmente
\usepackage{fancyhdr} %Para poder personalizar los encabezados y pies de las páginas
\usepackage{lastpage} %Para poder acceder al número final de páginas 
\lhead{Curso de Introducción a \LaTeX} %Encabezado izquierdo
\rhead{CICIMAR-IPN} %Encabezado derecho
\cfoot{} %Sin paginación al centro del pie de página
\rfoot{Página \thepage\ de \pageref{LastPage}} %Escribir la paginación a la izquierda dentro del pie de página incluyendo el número de la última página
\begin{document}
\frontmatter
\maketitle
\chapter*{Lorem ipsum dolor}
\lipsum
\tableofcontents
\pagestyle{fancy} %Establecer el estilo de página a fancy para que aplique los estilos personalizados a partir de aquí
\mainmatter
\chapter{Volutpat leo habitant}
% Debido a que el comando chapter reestablece el estilo de paginación y encabezados por defecto, hay que indicar que utilice el estilo fancy:
\thispagestyle{fancy}
\lipsum[1]
\section{Lorem ipsum}
\lipsum
\chapter{Adipiscing diam euismod}
\thispagestyle{fancy} % Repetir para cada capítulo (no es necesario con las secciones)
\lipsum[2]
\section{Phasellus}
\lipsum
\end{document}