\documentclass[10pt,letterpaper]{article}
\usepackage[utf8]{inputenc}
\usepackage{enumerate} %Para personalizar la apariencia de la enumeración
\title{Listas personalizadas}
\author{Curso de introducción a LaTeX}
\begin{document}
\maketitle

Podemos cambiar el icono mostrado en los elementos de una lista simple con la opción del comando \textbackslash\texttt{item}:

\begin{itemize}
\item[*]  Números complejos
\item[+] Funciones analíticas
\item Integración compleja
\end{itemize}

Si queremos modificar todos los iconos por una nueva imagen de una sola vez, podemos utilizar el comando \textbackslash\texttt{renewcommand}:

\renewcommand{\labelitemi}{$\rightarrow$}
\begin{itemize}
\item Números complejos
\item Funciones analíticas
\item Integración compleja
\end{itemize}

Finalmente, para modificar la forma en que \LaTeX enumera las listas, podemos utilizar la opción del entorno \texttt{enumerate} (gracias al paquete \texttt{enumerate}), por ejemplo para mostrar una enumeración con los temas de un curso:

\section{Números complejos}
Los contenidos que se imparten en la parte de  variable
compleja son:

\begin{enumerate}[\bfseries Tem{a} 1 \ --- ]
  \item Números complejos
     \begin{itemize}
       \item \emph{Construcción de los números complejos.}
       \item Forma polar. \textsf{Forma exponencial.}
     \end{itemize}
  \item Funciones analíticas
     \begin{itemize}
       \item \textbf{Funciones de una variable compleja.}
       \item Ecuaciones de Cauchy-Riemann.
     \end{itemize}
  \item Integración compleja
     \begin{description}
       \item[Teorema de Cauchy--Goursat] 
        Demostración y ejemplos de aplicaciones
       \item[Primitivas e Independencia del camino]
        Aplicaciones para la teoría de polos y residuos
     \end{description}
\end{enumerate}
\end{document}
