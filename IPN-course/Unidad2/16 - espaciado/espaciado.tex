\documentclass[10pt,letterpaper]{article}
\usepackage[utf8]{inputenc}
\title{Espacio vertical y horizontal}
\author{Curso de introducción a LaTeX}
\begin{document}
\maketitle
\section{Anatomía}
\vspace{-0.5cm}

El peso y tamaño del lobo puede variar considerablemente a lo largo del mundo, y tiende a incrementarse proporcionalmente con la latitud, como predijo la regla de Bergmann. En términos generales la altura varía entre los 60 y los 90 centímetros hasta el hombro, y tienen un peso de entre 32 y 70 kilos. Aunque raramente encontrados, especímenes de más de 77 kg han sido hallados en Alaska y Canadá; el lobo salvaje más pesado, matado en Alaska en 1939, pesaba 80 kg. Hay algunos casos sin confirmar de lobos cazados en el nordeste de Rusia que alcanzaban los 100 kg. Los lobos más pequeños son las sub-especies de lobos árabes, las hembras de éstas pueden pesar unos 10 kg en la madurez. Las hembras en una población dada pesan alrededor de un 20\% menos que los machos. Los lobos pueden medir entre 1,3 a 2 metros desde el hocico hasta la punta de la cola, siendo ésta aproximadamente un cuarto de la longitud total del cuerpo.
\vspace{2cm}

Los lobos poseen rasgos ideales para viajes de larga distancia. Su estrecho pecho y su potente espalda y piernas facilitan una locomoción eficiente. Son capaces de cubrir varios kilómetros trotando a una velocidad de 10 km/h, pudiendo alcanzar velocidades punta de 65 km/h en una persecución. Mientras corren a gran velocidad pueden cubrir cinco metros por salto. Las patas de los lobos están diseñadas para andar con facilidad por una amplia variedad de terrenos, especialmente nieve. Tienen una pequeña membrana entre cada dedo, lo que les permite moverse por la nieve con más facilidad que a sus presas. Los lobos son digitígrados, y cuentan con patas traseras más largas y un quinto dedo vestigial, solo presentes en las delanteras, siendo sus garras de coloración oscura/negra y no retráctiles. Pelos erectos y garras desafiladas realzan el agarre en superficies resbaladizas, y vasos sanguíneos especiales evitan el enfriamiento de las almohadillas de las patas. Unas glándulas les ayudan a moverse por grandes extensiones mientras informa a los otros acerca de su paradero.
 
%Eliminar la sangría automática
\noindent El mayor tamaño y longitud de las patas, ojos amarillos y mayores dientes hacen distinguir a los lobos adultos de otros cánidos, particularmente perros. Existe una glándula odorífica presente en la base de la cola de los lobos, la cual le confiere a cada individuo un rastro aromático único, que les sirve para poder identificarse entre ellos.
 
\subsection{Dentadura}
Los lobos y la mayoría de los perros grandes \hspace{1cm} comparten idéntica dentadura; el maxilar tiene seis incisivos, dos caninos, ocho premolares y cuatro molares. El maxilar inferior tiene seis incisivos, dos caninos, ocho premolares y seis  molares.
 
Los cuatro premolares superiores y los primeros molares inferiores constituyen los dientes carnasiales, los cuales son herramientas esenciales para cortar carne. Los largos dientes caninos son también importantes, ya que están diseñados para mantener y contener a la presa. Por tanto, cualquier lesión en la mandíbula o en los dientes puede ser devastador para un lobo, destinándolo a la inanición o a la \hfill incapacidad.
 
\subsection{Pelaje}

En ocasiones un lobo parece más pesado de lo que realmente es, debido a su voluminoso pelaje, compuesto por dos capas. La primera capa está diseñada para repeler el agua y la suciedad. La segunda es un denso subpelaje resistente al agua que aísla al lobo. Éste se torna en una gran mata de pelo a finales de primavera o comienzos de verano. Un lobo se frota normalmente contra objetos tales como rocas y ramas para fomentar la pérdida del pelaje. El subpelaje es usualmente gris sin tener en cuenta la apariencia del pelaje exterior. Los lobos tienen distintos pelajes en invierno y en verano que alternan en primavera y otoño. Las hembras tienden a conservar sus pelajes invernales más allá de la primavera a diferencia de los \dotfill machos.
 
La coloración del pelaje  varía
\vfill
según la estación.
\end{document}