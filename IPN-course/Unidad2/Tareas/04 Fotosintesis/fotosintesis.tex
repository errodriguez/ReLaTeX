\documentclass[10pt,letterpaper]{article}
\usepackage[utf8]{inputenc}
\usepackage[spanish, mexico]{babel}
%\usepackage{fixltx2e} %Para soportar el uso de subíndices dentro del texto
\usepackage{xcolor} %Para colorear el texto resaltado
\usepackage{soul}
%\usepackage{soulutf8} %Para soportar el texto subrayado y resaltado
\usepackage{url} %Para darle formato al enlace final
\author{Tu nombre}
\parskip=2mm

\newcommand{\highlight}[1] {\colorbox{yellow}{#1}}


\begin{document}
	\begin{center}
\underline{{\huge Fotosíntesis}}

Eduardo René Rodríguez Ávila
\end{center}

La \textbf{fotosíntesis}, del griego antiguo \textit{fos-fotós}, ``luz'', y \textit{sýnthesis} ``composición'', ``síntesis'', es la base de la mayor parte de la vida actual en la Tierra. Proceso mediante el cual las plantas, algas y algunas bacterias captan y utilizan la energía de la luz para transformar la materia inorgánica de su medio externo en materia orgánica que utilizarán para su crecimiento y desarrollo.

Los organismos capaces de llevar a cabo este proceso se denominan \textbf{fo\-to\-au\-tó\-tro\-fos} y además son capaces de fijar el CO$_{2}$ atmosférico (lo que ocurre casi siempre) o simplemente autótrofos. Salvo en algunas bacterias, en el proceso de fotosíntesis se producen liberación de oxígeno molecular (proveniente de moléculas de H\textsubscript{2}O) hacia la atmósfera (\textit{fotosíntesis oxigénica}). Es ampliamente admitido que el contenido actual de oxígeno en la atmósfera se ha generado a partir de la aparición y actividad de dichos organismos fotosintéticos. \ul{Esto ha permitido la aparición evolutiva y el desarrollo de organismos aerobios capaces de mantener una alta tasa metabólica} \hl{(el metabolismo aerobio es muy eficaz desde el punto de vista energético)}.

La otra modalidad de fotosíntesis, la \ul{fotosíntesis anoxigénica}, en la cual no se libera oxígeno, es llevada a cabo por un número reducido de bacterias, como las bacterias púrpuras del azufre y las bacterias verdes del azufre; estas bacterias usan como donador de hidrógenos el H$_2$S, con lo que liberan azufre.

Más información en: \url{https://es.wikipedia.org/wiki/Fotosintesis}
\end{document}