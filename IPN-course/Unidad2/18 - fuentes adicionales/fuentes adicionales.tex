\documentclass[12pt,letterpaper]{article}
\usepackage[utf8]{inputenc}
\usepackage{times} %Cambiamos la fuente de todo el documento a una fuente postscript (en este caso times). Para cada una de estas fuentes existe un paquete que la soporta, otros paquetes disponibles para usarse directamente son: pslatex, newcent, utopia o palatino. El catálogo completo de paquetes de fuentes aparece en el enlace al final del documento.
%\usepackage{palatino}
\usepackage{hyperref} %Para soportar el uso de hyperenlaces
\title{Fuentes externas}
\author{Introducción a LaTeX}
\begin{document}
%El título y el cuerpo se imprimirá en Times:
\maketitle

Cuerpo del texto (en Times):

- Lorem ipsum dolor sit amet, consectitur adipiscing elit.

%Si introducimos un texto con \textsf se imprimirá en Helvetica
Texto sin serifas (en Helvetica):

- \textsf{Lorem ipsum dolor sit amet, consectitur adipiscing elit.}

%Mientras que el texto obtenido con \texttt se imprimirá en Curier
Texto de máquina de escribir (en Courier):

- \texttt{Lorem ipsum dolor sit amet, consectitur adipiscing elit.}

\vspace{0.3cm}

Helvetica aplicada directamente al texto:

\fontfamily{phv}
\selectfont
- Lorem ipsum dolor sit amet, consectitur adipiscing elit.


Palatino aplicada directamente al texto: 

\fontfamily{ppl}
\selectfont
- Lorem ipsum dolor sit amet, consectitur adipiscing elit.

\vspace{0.3cm}

Fuentes originales:
%Computer Modern Roman:
\fontfamily{cmr}
\selectfont
- {Lorem ipsum dolor sit amet, consectitur adipiscing elit.}

%Y con la fuente original Sans Serif:
\fontfamily{cmss}
\selectfont
- {Lorem ipsum dolor sit amet, consectitur adipiscing elit.}

Para ver la lista completa de fuentes soportados por \fontfamily consultar la tabla del bloc de notas de la clase o en los documentos de la bibliografía

\vspace{0.3cm}

Para revisar el catálogo completo de paquetes de fuentes de \LaTeX \ visita: 

\url{https://tug.org/FontCatalogue/}
\end{document}