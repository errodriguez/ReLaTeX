\documentclass[12pt,letterpaper]{article}
\usepackage[utf8]{inputenc}
\usepackage{chemfig}
\usepackage[hidelinks]{hyperref}
\author{Curso de \LaTeX}
\title{Opciones de los enlaces químicos}
\begin{document}
\maketitle

Para representar múltiples enlaces alrededor de un átomo, utilizamos un paréntesis para indicar cada nuevo enlace:

\begin{center}
\chemfig{C
	(-[1]1)
	(-[2]2)
	(-[3]3)
	(-[4]4)
	(-[5]5)
	(-[6]6)
	(-[7]7)-0}
\end{center}

En el ejemplo anterior utilizamos ángulos predefinidos, y nos es de gran ayuda para visualizar los 8 posibles ángulos predefinidos que podemos utilizar.
 
De esta forma podemos representar una molécula de metano\footnote{\url{http://es.wikipedia.org/wiki/Metano}} utilizando ángulos predefinidos:

\begin{center}
\chemfig{C(-[2]H)(-[4]H)(-[6]H)-H}
\end{center}

La molecula anterior también se podría escribir con ángulos absolutos de la siguiente forma:

\begin{center}
\chemfig{C(-[:0]H)(-[:90]H)(-[:180]H)(-[:270]H)}
\end{center}

Puede haber varias submoléculas que estén atadas a un mismo átomo. Para representarlas, solo tenemos que escribir varios paréntesis que contengan el código de cada submolécula:

\begin{center}
\chemfig{A-B(-[1]W-X)(-[6]Y-[7]Z)-C}
\end{center}
\end{document}
