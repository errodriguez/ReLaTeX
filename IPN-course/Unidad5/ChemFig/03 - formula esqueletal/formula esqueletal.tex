\documentclass[12pt,letterpaper]{article}
\usepackage[utf8]{inputenc}
\usepackage{chemfig}
\usepackage[hidelinks]{hyperref}
\author{Curso de \LaTeX}
\title{Fórmula esqueletal}
\begin{document}
\maketitle
Una fórmula esqueletal  básica\footnote{\url{http://es.wikipedia.org/wiki/F\%C3\%B3rmula_esqueletal}} puede producirse de la siguiente forma: 

\begin{center}
\chemfig{-[:30]-[:-30]-[:30]-[:-30]}
	
\chemfig{-[:30]=[:-30]-[:30]}

\chemfig{-[:40]-[:0]~[:0]-[:0]-[:30]}
\end{center}
\end{document}
