\documentclass[12pt,letterpaper]{article}
\usepackage[utf8]{inputenc}
\usepackage{chemfig}
\author{Curso de \LaTeX}
\title{Estructuras químicas}
\begin{document}
\maketitle

\texttt{chemfig} es un paquete utilizado para dibujar gráficos químicos en 2D, basado en el paquete tikz, diseñado específicamente para representar expresiones químicas.

\section{Uso básico}

El primer comando que debemos aprender a usar con este paquete es \textbackslash\texttt{chemfig}:

\begin{center}
\chemfig{C-H}
\end{center}

El argumento que recibe este comando se subdivide en tres partes: el primer átomo, el tipo de enlace y el segundo átomo.

Hay nueve tipos de enlaces soportados:

\begin{center}
\chemfig{A-B}\\
\chemfig{A=B}\\
\chemfig{A~B}\\
\chemfig{A>B}\\
\chemfig{A<B}\\
\chemfig{A>:B}\\
\chemfig{A<:B}\\
\chemfig{A>|B}\\
\chemfig{A<|B}
\end{center}

Podemos escribir varios enlaces de manera secuencial:

\begin{center}
\chemfig{H-C=S}
\end{center}

\section{Ángulo de los enlaces}

Un enlace puede recibir una o más opciones entre corchetes, siendo la primera de ellas el ángulo que tendrá el enlace con respecto al último átomo.

Hay 3 tipos de ángulos que podemos definir: absolutos, relativos y predefinidos. Los absolutos indican un ángulo preciso (generalmente de 0 a 360, aunque también pueden ser negativos) y son representados con la sintaxis [:$ < $ángulo absoluto$ > $]. Por ejemplo:

\begin{center}
\chemfig{C-[:45]H}
\end{center}

Los ángulos relativos requieren la sintaxis  [::$ < $ángulo relativo$ > $] y producen un ángulo relativo al del enlace anterior. Por ejemplo:

\begin{center}
\chemfig{H-[:45]C-[::45]H}
\end{center}

Finalmente, los ángulos predefinidos son números enteros que van del 0 al 7 indicando intervalos de 45°:

\begin{center}
\chemfig{C-[1]H}

\chemfig{A-B-[1]C-[3]-D}
\end{center}

\section{Coeficiente del enlace}

La segunda opción que puede recibir un enlace, es el coeficiente del enlace:

\begin{center}
 \chemfig{H-C=[:45, 1.5]S}
\end{center}

Con el cual describimos el tamaño del enlace.

\section{Átomos de salida y arribo }

La tercer y cuarta opción del ángulo se utiliza cuando agrupamos varios átomos en un solo elemento de la estructura química y deseamos controlar qué átomos serán los que se conectaran. Por ejemplo, si consideramos los siguientes conjuntos:

\begin{flushleft}
\chemfig{ABCD-[:75]EFG}\quad
\chemfig{ABCD-[:-85]EFG}\quad
\chemfig{ABCD-[1]EFG}
\end{flushleft}

Veremos que \LaTeX siempre utiliza el último y primer átomo de cada conjunto para realizar el enlace. Utilizando las opciones del enlace que mencionamos, podemos cambiar este comportamiento:

\begin{flushleft}
\chemfig{ABCD-[:75,,2,3]EFG}\qquad
\chemfig{ABCD-[:75,,,2]EFG}\qquad
\chemfig{ABCD-[:75,,3,2]EFG}
\end{flushleft}

\section{Código \texttt{tikz}}

Finalmente, hay una quinta opción que podemos escribir con la que especificamos las características del gráfico dibujado con código del paquete \texttt{tikz}:

\begin{center}
 \chemfig{H-C=[:45, 1.2,,, draw=none]S}
 
 \chemfig{A-[,,,,red]B}
 
 \chemfig{A-[,,,,dash pattern=on 4pt off 2pt]B}
 
 \chemfig{A-[,,,,line width=2pt]B}
 
 \chemfig{A-[,,,,red,line width=2pt]B}
\end{center}

\end{document}
