\documentclass[12pt,letterpaper]{article}
\usepackage[utf8]{inputenc}
\usepackage{chemfig} %Para elaborar estructuras químicas
\usepackage[spanish, mexico]{babel} %Para que el contenido del documento esté en español de México
\usepackage[hidelinks]{hyperref} %Para el enlace al final del documento

% Para que funcione las excepciones del silabeo
\usepackage[T1]{fontenc}
\hyphenation{ha-ló-ge-nos or-gá-ni-cos com-pues-tos fun-da-men-tal-men-te sul-fa-to in-te-rés re-pe-ti-ción fos-fo-di-és-ter nu-clei-cos hi-dró-ge-no in-for-ma-ción mi-ne-ral}


% Preamble
\title{Compuesto orgánico}
\author{Curso de \LaTeX}


\begin{document}

\maketitle

Compuesto orgánico o molécula orgánica es una sustancia química que contiene carbono, formando enlaces carbono-carbono y carbono-hidrógeno. En muchos casos contienen oxígeno, nitrógeno, azufre, fósforo, boro, halógenos y otros elementos menos frecuentes en su estado natural. Estos com\-pues\-tos se denominan moléculas or\-gá\-ni\-cas. Algunos com\-pues\-tos del carbono, carburos, los carbonatos y los óxidos de carbono, no son moléculas or\-gá\-ni\-cas. La principal característica de estas sustancias es que arden y pueden ser quemadas (son compuestos combustibles). La mayoría de los compuestos orgánicos se producen de forma artificial mediante síntesis química aunque algunos todavía se extraen de fuentes naturales.


\begin{figure}[h]
	\begin{center}
		\centering
		\setchemfig{atom sep=2.0em}
		\chemfig{
			C
			(-[:90]H)
			(-[:225]H)
			(<[:270]H)
			(<:[:315]H)    	    	
		}
	\end{center}
	\vspace{-5mm}
	\caption{fórmula estructural del metano, un alcano y el compuesto orgánico más simple.}
\end{figure}


Las moléculas orgánicas pueden ser de dos tipos:

\begin{itemize}
	\item 	Moléculas orgánicas naturales: son las sintetizadas por los seres vivos, y se llaman biomoléculas, las cuales son estudiadas por la bioquímica y las derivadas del petróleo como los hidrocarburos.
	\item 	Moléculas orgánicas artificiales: son sustancias que no existen en la naturaleza y han sido fabricadas o sintetizadas por el hombre, por ejemplo los plásticos.
\end{itemize}




La línea que divide las moléculas orgánicas de las inorgánicas ha originado polémicas e históricamente ha sido arbitraria, pero generalmente, los compuestos orgánicos tienen carbono con enlaces de hidrógeno, y los compuestos inorgánicos, no. Así el ácido carbónico es inorgánico, mientras que el ácido fórmico, el primer ácido carboxilico, es orgánico. El anhídrido carbónico y el monóxido de carbono, son compuestos inorgánicos. Por lo tanto, todas las moléculas orgánicas contienen carbono, pero no todas las moléculas que contienen carbono son moléculas orgánicas.

\section{Historia}

La etimología de la palabra ``orgánico'' significa que procede de órganos, relacionado con la vida; en oposición a ``inorgánico'', que sería el calificativo asignado a todo lo que carece de vida. Se les dio el nombre de orgánicos en el siglo XIX, por la creencia de que sólo podrían ser sintetizados por organismos vivos. La teoría de que los compuestos orgánicos eran fundamentalmente diferentes de los ``inorgánicos'', fue refutada con la síntesis de la urea, un compuesto "orgánico" por definición ya que se encuentra en la orina de organismos vivos, síntesis realizada a partir de cianato de potasio y sulfato de amonio por Friedrich Wöhler (síntesis de Wöhler). Los compuestos del carbono que todavía se consideran inorgánicos son los que ya lo eran antes del tiempo de Wöhler; es decir, los que se encontraron a partir de fuentes sin vida, ``inorgánicas'', tales como minerales.

\section{Clasificación según su origen}

La clasificación por el origen suele englobarse en dos tipos: natural o sintético.

\subsection{Natural}

\subsubsection{In-vivo}

Los compuestos orgánicos presentes en los seres vivos o ``biosintetizados'' constituyen una gran familia de compuestos orgánicos. Su estudio tiene interés en bioquímica, medicina, farmacia, perfumería, cocina y muchos otros campos más.

\noindent \textbf{Carbohidratos} \quad Los carbohidratos están compuestos fundamentalmente de carbono (C), oxígeno (O) e hidrógeno (H). Son a menudo llamados ``azúcares'' pero esta nomenclatura no es del todo correcta. Tienen una gran presencia en el reino vegetal (fructosa, celulosa, almidón, alginatos), pero también en el animal (glucógeno, glucosa). Se suelen clasificar según su grado de polimerización en:

\begin{itemize}
	\item Monosacáridos (fructosa, ribosa y desoxirrobosa)
	\item Disacáridos (sacarosa, lactosa)
	\item Trisacáridos (maltotriosa, rafinosa)
	\item Polisacáridos (alginatos, ácido algínico, celulosa, almidón, etc)
	
\end{itemize}

\noindent \textbf{Lípidos} \quad Los lípidos son un conjunto de molécula orgánicas, la mayoría biomoléculas, compuestas principalmente por carbono e hidrógeno y en menor medida oxígeno, aunque también pueden contener fósforo, azufre y nitrógeno. Tienen como característica principal el ser hidrófobas (insolubles en agua) y solubles en disolventes orgánicos como la bencina, el benceno y el cloroformo. En el uso coloquial, a los lípidos se les llama incorrectamente grasas, ya que las grasas son sólo un tipo de lípidos procedentes de animales. Los lípidos cumplen funciones diversas en los organismos vivientes, entre ellas la de reserva energética (como los triglicéridos), la estructural (como los fosfolípidos de las bicapas) y la reguladora (como las hormonas esteroides).
\\

\noindent \textbf{Proteínas} \quad Las proteínas son polipéptidos, es decir están formados por la polimerización de péptidos, y estos por la unión de aminoácidos. Pueden considerarse así "poliamidas naturales" ya que el enlace peptídico es análogo al enlace amida. Comprenden una familia importantísima de moléculas en los seres vivos pero en especial en el reino animal. Ejemplos de proteínas son el colágeno, las fibroinas, o la seda de araña.
\\

\noindent \textbf{Ácidos nucleicos} \quad Los ácidos nucleicos son polímeros formados por la repetición de monómeros denominados nucleótidos, unidos mediante enlaces fosfodiéster. Se forman, así, largas cadenas; algunas moléculas de ácidos nucleicos llegan a alcanzar pesos moleculares gigantescos, con millones de nucleótidos encadenados. Están formados por las partículas de carbono, hidrógeno, oxígeno, nitrógeno y fosfato.Los ácidos nucleicos almacenan la información genética de los organismos vivos y son los responsables de la transmisión hereditaria. Existen dos tipos básicos, el ADN y el ARN.

\begin{figure}[h]
	\centering
	\chemfig{O=[:35]*6(-=(*6(---(*6(-(*5(---(<OH)--))-----))---))----)}
	\caption{Estructura de la testosterona. Una hormona, que se puede clasificar como ``molécula pequeña'' en el argot químico-orgánico.}
\end{figure}

\noindent \textbf{Moléculas pequeñas} \quad Las moléculas pequeñas son compuestos orgánicos de peso molecular moderado (generalmente se consideran "pequeñas" aquellas con peso molecular menor a 1000 g/mol) y que aparecen en pequeñas can\-ti\-da\-des en los seres vivos pero no por ello su importancia es menor. A ellas pertenecen distintos grupos de hormonas como la testosterona, el estrógeno u otros grupos como los alcaloides. Las moléculas pequeñas tienen gran interés en la industria farmacéutica por su relevancia en el campo de la medicina.

\subsubsection{Ex-vivo}

Son compuestos orgánicos que han sido sintetizados sin la intervención de ningún ser vivo, en ambientes extracelulares y extravirales.
\\

\noindent \textbf{Procesos geológicos} \quad El petróleo es una sustancia clasificada como mineral en la cual se presentan una gran cantidad de compuestos orgánicos. Muchos de ellos, como el benceno, son empleados por el hombre tal cual, pe\-ro muchos otros son tratados o derivados para conseguir una gran cantidad de compuestos orgánicos, como por ejemplo los monómeros para la síntesis de materiales poliméricos o plásticos.
\begin{figure}[h]
	\centering
	\chemfig{H-[:30]*6(=(-H)-(-H)=(-H)-(-H)=(-H)-)}
	\caption{estructura del benceno}
\end{figure}

\subsubsection*{Procesos atmosféricos}

\noindent \textbf{Procesos de síntesis planetaria} \quad En el año 2000 el ácido fórmico, un compuesto orgánico sencillo, también fue hallado en la cola del cometa Hale-Bopp. Puesto que la síntesis orgánica de estas moléculas es inviable bajo las condiciones espaciales este hallazgo parece sugerir que a la formación del sistema solar debió anteceder un periodo de calentamiento durante su colapso final.

\subsection{Sintético}

Desde la síntesis de Wöhler de la urea un altísimo número de compuestos orgánicos han sido sintetizados químicamente para beneficio humano. Estos incluyen fármacos, desodorantes, perfumes, detergentes, jabones, fibras téxtiles sintéticas, materiales plásticos, polímeros en general, o colorantes orgánicos.

Más información en: \url{http://es.wikipedia.org/wiki/Compuesto_org\%C3\%A1nico}
\end{document} 
