\documentclass[10pt,letterpaper]{article}
\usepackage[utf8]{inputenc}
\usepackage{colortbl} %Para poder utilizar el comando \rowcolor
\usepackage{multirow} %Para combinar múltiples renglones
\usepackage{hyperref} %Para soportar hyperenlaces (URLs)
\usepackage{rotating} %Para soportar el comando \rotatebox
\title{Más efectos sobre las tablas}
\author{Curso de introducción a LaTeX}
\begin{document}
Colores en tablas:\\
\definecolor{LightBlue}{rgb}{0.8,0.85,1}
\begin{tabular}{ccc}
    \rowcolor{LightBlue} TextoCelda1 & TextoCelda2 & TextoCelda3 \\
    TextoCelda4 & TextoCelda5 & TextoCelda6 \\
    TextoCelda7 & TextoCelda8 & TextoCelda9
\end{tabular}
%En este ejemplo, definimos un nuevo color LightBlue y se lo asignamos como color de fondo a la primera fila. 

\vspace{1cm}

Columnas de extensión múltiple (combinar celdas):\\
\begin{tabular}{ |l|l| }
  \hline
  \multicolumn{2}{|c|}{Tabla de equipos} \\
  \hline
  GK & Paul Robinson \\
  LB & Lucus Radebe \\
  DC & Michael Duberry \\
  DC & Dominic Matteo \\
  RB & Dider Domi \\
  MC & David Batty \\
  MC & Eirik Bakke \\
  MC & Jody Morris \\
  FW & Jamie McMaster \\
  ST & Alan Smith \\
  ST & Mark Viduka \\
  \hline
\end{tabular}
%En éste ejemplo, definimos que el primer renglon ocupará 2 columnas de la tabla, con su propia alineación al centro, ignorando la alineación definida más arriba

\vspace{1cm}

Combinar celdas en varios renglones:\\
\begin{tabular}{ccccc}
	\hline
	& \textbf{Sexo} & \textbf{Grupo} & \textbf{Estatura} & \textbf{Peso}\\
	\hline
	\multirow{6}{*}{\rotatebox{90}{\textbf{Observaciones }}} & F & Adulto & 1.6 & 60\\
	& M & Menor & 1.3 & 45\\
	& M & Adulto & 1.7 & 85\\
	& F & Adulto & 1.5 & 74\\
	& M & Adulto & 1.6 & 77\\
	& F & Anciano & .5 & 65\\
	& F & Menor & 1.2 & 43\\
	\hline
\end{tabular}
%En este tercer ejemplo, definimos que la primera columna del segundo renglon ocupará seis renglones de la tabla y tendrá un ancho natural (ajustándose al ancho del contenido). El texto aparece rotado verticalmente con el comando \rotatebox, que nos permite rotar verticualmente cualquier elemento del documento

\vspace{1cm}

Definiendo varias columnas de un golpe:\\
\begin{tabular}{l*{6}{c}r}
	Equipo           & J & G & E & P & F  & C & Pts \\
	\hline
	Manchester United & 6 & 4 & 0 & 2 & 10 & 5 & 12  \\
	Celtic            & 6 & 3 & 0 & 3 &  8 & 9 &  9  \\
	Benfica           & 6 & 2 & 1 & 3 &  7 & 8 &  7  \\
	FC Copenhagen     & 6 & 2 & 1 & 3 &  5 & 8 &  7  \\
\end{tabular}
%En este último ejemplo definimos varias columnas con una sola instrucción: *{6}{c}, lo cual significa que habrá 6 columnas con alineación al centro, antecedidas por una columna con alineación a la izquierda y seguidas de una columna con alineación a la derecha

\vspace{1cm}

También podemos generar tablas de datos más fácilmente utilizando la he\-rra\-mienta \textbf{LaTeX Table Generator} en el sitio: \url{https://tablesgenerator.com/}, que incluye una función para transformar tablas desde Excel o Calc a \LaTeX (File $\rightarrow$ Paste table data...).

\end{document}